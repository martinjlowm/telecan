\chapter{Related Work}
In this chapter, we take a brief look at other existing software
defined radio solutions and projects that have inspired my process,
design and implemention of this project.
\begin{itemize}
\item GNU Radio is a free and open-source software development toolkit
  that provides signal processing blocks to implement software
  radios. It can be used with readily-available low-cost external RF
  hardware to create software-defined radios, or without hardware in a
  simulation-like environment. It is widely used in hobbyist, academic
  and commercial environments to support both wireless communications
  research and real-world radio systems~\cite{gnuradio}.
\item AirProbe is a \gls{GSM}-sniffer project that provides tools to
  parse \gls{GSM} communication using \gls{SDR}.\ Its main module
  interfaces with GNU Radio from which the tuning frequency and gain
  can be changed through a \gls{GUI}~\cite{airprobe}.
\item gr-gsm is an actively developed project based on the \gls{GSM}
  receiver module from AirProbe. It provides the same functionality as
  AirProbe, thus decodes \gls{GSM} communication using GNU Radio data
  structures and graphical interface~\cite{grgsm}.
\item Kalibrate is a utility, originally developed by Joshua Lackey,
  that scans \gls{GSM} bands for base transceiver stations by
  searching for a frequency correction burst that is transmitted
  periodically by the BTS. It uses the algorithm as described in the
  paper \textit{Robust Frequency Burst Detection Algorithm for GSM /
    GPRS} by G. Narendra Varma, Usha Sahu and G. Prabhu Charan to find
  a pure sinusoid of all zero bits within a GSM
  burst~\cite{kalibrate}.
\item OpenBSC is an open source Linux software application that
  implements \gls{GSM} telecommunication protocols and functions as a
  software-based mobile backhaul entity using a software defined radio
  as transceiver peripheral~\cite{openbsc}.
\item OpenBTS is similar to OpenBSC, but is not limited to \gls{GSM}
  protocols and it supports mobile devices presented as \gls{SIP},
  endpoints which makes \gls{VoIP}, possible~\cite{openbts}.
\item gr-LTE provides a similar approach of gr-gsm by demodulating a
  signal and then synchronizing to it, but for \gls{LTE}
  networks~\cite{grlte}.
\end{itemize}
