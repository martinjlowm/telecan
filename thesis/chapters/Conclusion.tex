\chapter{Conclusion}
This thesis proposes a protocol analysation solution for mobile
communication technologies that analyses the underlying protocols used
on the air interface to control and communicate with mobile entities
wirelessly. This solution supports parts of the \gls{GSM} standard as
specified by the \gls{3GPP} and allows a synchronous link between the
\gls{BTS} and connected \glspl{MS} in a specific cell of interest. The
Um interface consists of protocols which enable a solid wireless
connection that introduces multiplexing principles of \gls{TDMA} and
\gls{FDMA} to allow multiple parallel connections of several bursts
carried in timeslots across multiple frequencies.

Through analysis of these bursts, the underlying protocols of
\gls{GSM} can be examined and this examination shows the
\gls{ISDN}-derived \gls{LAPDm} structured data that controls the layer
three protocols of \gls{GSM} for call-, mobility- and radio
control. The radio link introduces modulation techniques for
transmission of these bursts and their digital information through
phase manipulation of radio waves. This enables the use of \gls{SDR}
and its ability to adapt to modulation schemes as specified by the
\gls{GSM} standard. \gls{SDR} provides a flexible platform for
development of software-based wireless solutions and mobile
technologies are no exception.

By defining an \gls{SDR} through software for telecommunication
purposes, mobile technologies become more accessible and their
underlying use of protocols easier to learn and comprehend.

The protocol analyzer design and implementation described in this
thesis is the beginning of a contributed tool that is freely available
and is proposed a use for education purposes in comprehension of
protocols in mobile communications.
