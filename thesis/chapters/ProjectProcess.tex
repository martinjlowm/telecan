\chapter{Project process}
This chapter walks through my point of view of the overall process
during the project and it reflects thereon. The inception phase of the
project started in February with an initial interest in the
understanding of the lower level protocols in mobile communications. I
chose \gls{GSM} as the main series of technologies to examine because
there is a safe and sufficient amount of documentation available. The
preliminary planning of the project was divided into multiple
milestones and a risk analysis was performed to predict possible
drawbacks.

\section{Preliminary planning}
The initial plan for the project was to divide the project into the
following milestones over the project period of twentyone weeks.

\begin{itemize}
\item Preparation of the project
\item Initial \gls{GSM} work using a HackRF One
\item Implement first wave of \gls{GSM} test cases
\item Implement second wave of \gls{GSM} test cases
\item Implement third wave of \gls{GSM} test cases
\item \gls{GUI} display of \gls{GSM} monitored information
\item Finalize project
\end{itemize}

The preparation of the project included a risk analysis and a search
for litterature that describes parts on the subject of communication
on the \gls{GSM} Um interface. With a collection of litterature at
hand, the initial design and implementation stage began. This led to a
series of waves to implement test cases relevant for parsing of
\gls{GSM} traffic. Lastly, the monitored traffic was planned to be
presented in a \gls{GUI} in time for the last milestone, such that the
focus could be on the thesis then.

For this process to be as smooth as possible the following risk
analysis was performed.

\subsection{Risk analysis}
\cref{tab:risk_analysis} lists my assumptions of possible drawbacks
that could occur during the project process.
\begin{table}[H]
  \centering
  \begin{tabular}{|p{0.15\linewidth}|
                   p{0.14\linewidth}|
                   p{0.1\linewidth}|
                   p{0.15\linewidth}|
                   p{0.30\linewidth}|}
    % BEGIN RECEIVE ORGTBL risk_analysis
    \hline
    \textbf{Description} & \textbf{Likelihood} & \textbf{Impact} & \textbf{Result} & \textbf{Prevention} \\
    \hline
    Writing the report could take longer than expected.
                & Medium     & Medium & The report will be handed in
                                        incomplete.
                                               & Multiple draft
                                                 reports should be
                                                 handed in during the
                                                 last milestone. Some
                                                 sections of the
                                                 report can be written
                                                 in parallel to the
                                                 coding in previous
                                                 milestones. \\
    \hline
    Time consumption of each milestones may be wrongly estimated.
                & High       & Medium & What was meant to be done in a
                                        specific milestone may not be
                                        complete.
                                               & Continue to the next
                                                 milestone and if time
                                                 allows it return back
                                                 to the incomplete one
                                                 and finish it. \\
    \hline
    Illness     & Low        & Medium & Tasks may be more time
                                        consuming.
                                               & None. \\
    \hline
    Physical damage to- or theft of laptop.
                & Low        & High   & Loss of work that in turn
                                        complicates the time plan
                                        schedule.
                                               & Have a cloud backup
                                                 service active at all
                                                 times. \\
    \hline
    Delay in HackRF One delivery.
                & High       & High   & Unable to implement software.
                                               & Look for an
                                                 alternative device to
                                                 use temporarily until
                                                 the HackRF One
                                                 arrives. \\
    \hline
    % END RECEIVE ORGTBL risk_analysis
  \end{tabular}
  \caption{Risk analysis made at project inception.}
  \label{tab:risk_analysis}
\end{table}
\begin{comment}
  #+ORGTBL: SEND risk_analysis orgtbl-to-latex :splice t
  |-------------+------------+--------+-------------+-------------------------|
  | <11>        | <10>       | <6>    | <11>        | <23>                    |
  | Description | Likelihood | Impact | Result      | Prevention              |
  |-------------+------------+--------+-------------+-------------------------|
  | Writing the report could take longer than expected. | Medium     | Medium | The report will be handed in incomplete. | Multiple draft reports should be handed in during the last milestone. Some sections of the report can be written in parallel to the coding in previous milestones. |
  |-------------+------------+--------+-------------+-------------------------|
  | Time consumption of each milestones may be wrongly estimated. | High       | Medium | What was meant to be done in a specific milestone may not be complete. | Continue to the next milestone and if time allows it return back to the incomplete one and finish it. |
  |-------------+------------+--------+-------------+-------------------------|
  | Illness     | Low        | Medium | Tasks may be more time consuming. | None                    |
  |-------------+------------+--------+-------------+-------------------------|
  | Physical damage to- or theft of laptop. | Low        | High   | Loss of work that in turn complicates the time plan schedule. | Have a cloud backup service active at all times. |
  |-------------+------------+--------+-------------+-------------------------|
  | Delay in HackRF One delivery. | High       | High   | Unable to implement software. | Look for an alternative device to use temporarily until the HackRF One arrives. |
  |-------------+------------+--------+-------------+-------------------------|
\end{comment}

\section{Evaluation}
The initial planning turned out to be unrealistic and the scope of the
project was much different than what was predicted. The project
process made a turn towards a more digital signal processing-based
focus rather than the intended analysis of \gls{GSM} traffic. This
implied that the goal of a \gls{GUI} and a fully implementation of
\gls{GSM} traffic parsing were not reached. The synchronization,
sequence estimation and decoding introduced many more elements behind
the scene than what were expected, and in turn a lot more theory to be
examined and learned. These factors were the main causes for the
imperfect result and there is, as listed in \cref{cha:future_work},
plenty more work to be done in the future.

To that end, the project has not been managed well and the aim of the
project could have been clearer if more time and energy had been spent
on researching the topic. The planned deadlines throughout the process
were not met and work kept getting postponed which led to a huge
amount of work the last month. If the scope of the project had been
clearer, it might have been easier to comply with revised deadlines.

The planned milestones ended up being meaningless as the specifics of
the \gls{GSM} air interface were learned, and time allocations of
``test case waves'' were greatly misestimated. These allocations
assumed that everything prior to the parser would have been done and
that was not the case. To that end, the HackRF One arrived eight weeks
late and this time should have been better valued in terms of
research. The misestimated time allocations might have been discovered
earlier and then work could have been distributed better.

A lot of work was done the last month, but there were some moments
that consumed more time than others. Since the software implementation
was developed as a standalone application and without any useful
sample debugging utility, a lot of time was spent on debugging the
code. If GNU Radio had been used as a platform, then a great deal of
this time would have been saved, since it provides efficient tools to
visualize and debug input samples. In the end, a more complete
implementation could have been accomplished if GNU Radio had been used
as a platform.
