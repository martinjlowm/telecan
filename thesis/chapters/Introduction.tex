\chapter{Introduction}
The \glsentryfull{GSM}, has grown to be the leading mobile
communication standard in the world since first introduced in
1982. \gls{GSM} was initially developed as a wireless alternative to
the \gls{PSTN}, which allowed intercommunication in both
technologies. \gls{GSM} was later developed to support packet
switching as a \gls{GPRS}, meaning it had access to the
Internet. Circuit- and packet switched networks share a stack of
protocols used on the Um air interface between a \glsentryfull{BTS},
and the mobile phone. One of such protocols is the \gls{LAPDm}, used
in the data link layer.

Since the air interface is an open medium, it is possible to listen in
on communication between a mobile phone and the base station. Using a
\glsentryfull{SDR}, some \gls{GSM} communication can be decoded by
implementing the freely available \gls{GSM} standards through software
as specified by the \gls{3GPP}. For security reasons, dedicated user
data that are exchanged are encrypted with an A/5.1 stream cipher, but
unfortunately, such data can be decrypted
quickly~\cite{barkan}. Newer, faster and more secure technologies
exist today to counter this vulnerability, for example \gls{UMTS},
\gls{HSPA}, and \gls{LTE}.\ While \gls{LTE} and newer technologies are
slowly getting deployed globally to fulfill customer needs of
\gls{QoS}, \gls{GSM} remains the leading standard in mobile coverage
for now.

An \gls{SDR} is built with the purpose of handling radio signals in
software. This is done by having \gls{SDR}-hardware sample radio
signals and forward those samples to host memory of a computer. HackRF
One is such a peripheral that forwards samples over a \gls{USB}.\ The
\gls{USB} interface also allows the HackRF to be configured by
software. Examples of such configurable parameters are tune frequency,
sample rate and gain. Its \gls{USB} interface is operated using the
HackRF C library, libhackrf.

The prime motivation of this thesis, is to contribute a tool which can
be used for education purposes in understanding of- and learning about
mobile communications. \gls{SDR} provides the means to make this
possible on a low budget and the software behind \gls{SDR}, makes it a
flexible tool for many applications. \gls{GSM} is one example.

In this report, I examine the \gls{GSM} technology and describe my
design and implementation of a mobile network protocol analyzer piece
of software, in terms of displaying mobile communication in real time,
using a HackRF One, an \gls{SDR}.\ The protocol analyzer software is
developed with several mobile communications technologies in mind for
easy extension, but currently only supports \gls{GSM}.\ It
synchronizes with a near base station and decodes data sent over the
air in the downlink direction.

\section{Thesis outline}
The thesis is outlined as briefly described in the following.
\begin{itemize}
\item Chapter 2 dives into alternative solutions and related work that
  accomplish the same functionality as the design and implementation
  for parsing \gls{GSM} traffic proposed in this thesis.
\item The infrastructure behind \gls{GSM} is walked through in Chapter
  3 starting with a wide perspective of the infrastructure that then
  digs into the Um interface between the \gls{BTS} and the mobile
  entity, its design and its \gls{LAPDm} protocols.
\item By further describing the Um interface, Chapter 4 looks into the
  theory behind the processing of signals on this interface. The
  chapter briefly describes sampling of a signal and modulation
  schemes used on the Um interface.
\item Chapter 5 takes the two previous chapters and describes
  \gls{SDR} in general in respect to mobile communications with focus
  on \gls{GSM}.\
\item Chapter 6 describes my ideas for- and design of the mobile
  network protocol analyzer with the \gls{SDR} interaction and each
  operation following the flow of samples towards the parsing element
  in the application.
\item The specifics of the implementation are introduced in Chapter 7
  which follows a similar flow of Chapter 6.
\item Chapter 8 evaluates my implementation of the protocol analyzer
  and a few examples of \gls{GSM} broadcast traffic content are tested
  on a $1800\si{MHz}$ \gls{GSM} band.
\item Some improvements and ideas that didn't make it to the current
  implementation, but may be implemented in the future are discussed
  in Chapter 9.
\item Chapter 10 evaluates and discusses my process throughout the
  project.
\item Finally, Chapter 11 concludes on the project.
\end{itemize}